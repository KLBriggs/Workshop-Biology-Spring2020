% Options for packages loaded elsewhere
\PassOptionsToPackage{unicode}{hyperref}
\PassOptionsToPackage{hyphens}{url}
%
\documentclass[
]{article}
\usepackage{lmodern}
\usepackage{amssymb,amsmath}
\usepackage{ifxetex,ifluatex}
\ifnum 0\ifxetex 1\fi\ifluatex 1\fi=0 % if pdftex
  \usepackage[T1]{fontenc}
  \usepackage[utf8]{inputenc}
  \usepackage{textcomp} % provide euro and other symbols
\else % if luatex or xetex
  \usepackage{unicode-math}
  \defaultfontfeatures{Scale=MatchLowercase}
  \defaultfontfeatures[\rmfamily]{Ligatures=TeX,Scale=1}
\fi
% Use upquote if available, for straight quotes in verbatim environments
\IfFileExists{upquote.sty}{\usepackage{upquote}}{}
\IfFileExists{microtype.sty}{% use microtype if available
  \usepackage[]{microtype}
  \UseMicrotypeSet[protrusion]{basicmath} % disable protrusion for tt fonts
}{}
\makeatletter
\@ifundefined{KOMAClassName}{% if non-KOMA class
  \IfFileExists{parskip.sty}{%
    \usepackage{parskip}
  }{% else
    \setlength{\parindent}{0pt}
    \setlength{\parskip}{6pt plus 2pt minus 1pt}}
}{% if KOMA class
  \KOMAoptions{parskip=half}}
\makeatother
\usepackage{xcolor}
\IfFileExists{xurl.sty}{\usepackage{xurl}}{} % add URL line breaks if available
\IfFileExists{bookmark.sty}{\usepackage{bookmark}}{\usepackage{hyperref}}
\hypersetup{
  pdftitle={Calcium carbonate precipitation in Everglades periphyton: ecosystem drivers and consequences for a microbial process},
  pdfauthor={Kristin L. Briggs},
  hidelinks,
  pdfcreator={LaTeX via pandoc}}
\urlstyle{same} % disable monospaced font for URLs
\usepackage[margin=1in]{geometry}
\usepackage{graphicx,grffile}
\makeatletter
\def\maxwidth{\ifdim\Gin@nat@width>\linewidth\linewidth\else\Gin@nat@width\fi}
\def\maxheight{\ifdim\Gin@nat@height>\textheight\textheight\else\Gin@nat@height\fi}
\makeatother
% Scale images if necessary, so that they will not overflow the page
% margins by default, and it is still possible to overwrite the defaults
% using explicit options in \includegraphics[width, height, ...]{}
\setkeys{Gin}{width=\maxwidth,height=\maxheight,keepaspectratio}
% Set default figure placement to htbp
\makeatletter
\def\fps@figure{htbp}
\makeatother
\setlength{\emergencystretch}{3em} % prevent overfull lines
\providecommand{\tightlist}{%
  \setlength{\itemsep}{0pt}\setlength{\parskip}{0pt}}
\setcounter{secnumdepth}{-\maxdimen} % remove section numbering

\title{Calcium carbonate precipitation in Everglades periphyton: ecosystem
drivers and consequences for a microbial process}
\author{Kristin L. Briggs}
\date{1/10/2020}

\begin{document}
\maketitle

\hypertarget{research-statement}{%
\section{Research Statement}\label{research-statement}}

Microbial communities drive important ecosystem functions in many
terrestrial and aquatic habitats, so it is crucial to understand how
environmental shifts will influence them. Periphyton mats of the
Everglades are an example of a `foundational' microalgal community which
maintain the system's unique biogeochemistry and habitat structure. They
sequester large amounts of carbon into organic matter and calcium
carbonate (CaCO3), and thereby fuel food webs, regulate carbon fluxes,
and build marl soils. The widespread precipitation of CaCO3 is key to
periphyton mat functioning, and more broadly, is an essential process in
our geological history which built soils, sedimentary rock, along with
other important functions, yet the mechanisms driving precipitation are
not well understood. Periphyton is an important tool for monitoring
water quality and Everglades restoration, and we know this algal
community shifts to non-calcareous biofilms of reduced functionality in
response to environmental shifts like prolonged hydroperiods and
phosphorus enrichment. Since CaCO3 precipitation within periphyton mats
is influenced by water chemistry and hydrology, it is imperative to
understand how changes in these conditions will influence the system's
carbon sequestration and storage. The Everglades' water flow is heavily
managed and large-scale modifications are ongoing to restore some
pre-drainage flow to improve ecosystem functioning and ecological
integrity, but the impacts of these efforts are still being studied.
Evaluating how CaCO3 precipitation in Everglades periphyton on the small
and intermediate scale are influenced by ecosystem gradients may
elucidate drivers of Everglade's carbon budget, microbial carbonate
precipitation, and help us optimize ecosystem functioning through water
management. Employing manipulative experiments, field studies,
analytical methods for structural and chemical analyses, and molecular
techniques (e.g.~SEM EDS, RNA) will allow us to answer important
questions about how modified hydrology (eg. water depth, chemistry) will
impact CaCO3 precipitation and periphyton functioning.

\hypertarget{methods}{%
\section{Methods}\label{methods}}

Depending on the data that I use, which I am still working on compiling,
I would like to use some multivariable analyses which can compare the
influence of hydrological and water chemistry variables on calcium
carbonate precipitation in periphyton.

\hypertarget{data}{%
\section{Data}\label{data}}

As of now I'm not sure if I will have quantitative data (calcium
carbonate volume, community composition data) on this yet, but I may be
able to use qualitative data (periphyton type) from the field coupled
with environmental metadata (water depth, pH, etc.) for some real-data
analyses. Otherwise I will need to use other data.

\end{document}
